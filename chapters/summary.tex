%!TEX root = ../main.tex
\label{section:summary}

This master thesis discusses the concepts of moving regions in moving object databases and focuses on the implementation of fixed-shape moving regions in MobilityDB, an open-source moving object database built on top of PostgreSQL and PostGIS.

In Chapters 2 and 3, the systems used and the current state of the research on moving regions are both described to give enough context for a clear understanding of the following sections. We here make a distinction between deformable moving regions, fixed-shape moving regions, and between 2D and 3D moving regions. We then decide to focus on the practical implementation of fixed-shape moving regions in 2D.

In the next part (Chapter 4), we dive into the main contribution of this master thesis, which is developing a model for fixed-shape moving regions in MOD's, and implementing this model in practice in MobilityDB. For this, multiple types are added into MobilityDB: the \lstinline{tgeometry} type, which has four variants (\lstinline{tgeometryinst, tgeometryi, tgeometryseq} and \lstinline{tgeometrys}) depending on its duration, and the \lstinline{rtransform} type, which is used to represent an instant of a region more concisely than by using a polygon.

In Section \ref{section:general_functions}, a large set of functions that can be applied to the new \lstinline{tgeometry} type is listed and described. Most of these functions are similar to existing MobilityDB functions but are simply extended to allow arguments with the \lstinline{tgeometry} type.

During the implementation of these functions different algorithms have been developed, and the most interesting ones are described in the last section of Chapter 4, Section \ref{section:internal_functions}. The first algorithm is used to compute the bounding box of a moving region efficiently, although not optimally. The second one describes the normalization process that is applied to all MobilityDB sequences when constructed, and the algorithm to normalize sequences of regions is also explained in detail. Section \ref{section:standalone_inst} talks about a particularity of temporal geometries compared to all other temporal types of MobilityDB and talks about the addition of a small utility function that is needed to emulate the usual behaviour of temporal types when working with temporal geometries. Lastly, the algorithm used to compute the traversed area of a moving region, as well as possible improvements,  is described thoroughly, since this the result of this algorithm is used in various functions.

Due to their complexity, multiple aspects of this project have been researched without being effectively implemented into MobilityDB yet. One particular algorithm for the interpolation of rotations in 3D is already developed and tested outside MobilityDB and is thus also thoroughly described in Section \ref{section:quaternion_interpolation}. Other aspects and extensions of this project are also discussed in Chapter 5, and their implementation is left as future work.
%!TEX root = ../main.tex

\section{Temporal Geography Type}
\label{section:tgeography}

Until currently, every region was described using the PostGIS \lstinline{geometry(Polygon)} type. However, PostGIS also exposes a \lstinline{geography(Polygon)} type, which can be used to represent a polygon using \textit{geographic coordinates}. These coordinates are spherical coordinates expressed in angular units (degrees), and these can be used to more accurately represent a position on a sphere (such as the earth), without having to project them on a sphere. Calculations done using these coordinates are more complicated, since they must be calculated on a sphere. Implementing a temporal geography type to handle moving regions using geographic coordinates would thus require other functions than the ones used by the \lstinline+tgeometry+ type. 

Moving points have been implemented in MobilityDB for both the \lstinline{geometry(Point)} and \lstinline{geography(Point)} types, so doing the same for moving regions would be consistent with the current implementation, but this is left as future work.

\section{Unify \texttt{tgeompoint} and \texttt{tgeometry}}
\label{section:tgeompoint}

PostGIS uses a unified type \lstinline{geometry} for all types of geometries: \lstinline+Point+, \lstinline+Polygon+, \lstinline+LineString+, etc. Currently, the MobilityDB implementation uses different types to represent moving points (\lstinline{tgeompoint}) and moving regions (\lstinline{tgeometry}), which are not related in any way. To be as consistent with the PostGIS implementation as possible, it might be interesting to create a unified SQL type \lstinline{tgeometry}, which can represent either a moving region (\lstinline{tgeometry(Polygon)}) or a moving point (\lstinline{tgeometry(Point)}, and would be handled differently by the SQL functions.

This is completely doable, but requires a lot of code refactoring and modifications. Since moving points are the predominant us case, this unification should also not slow down any functions handling moving points. Currently, since the only two types that could be unified are \lstinline{tgeompoint} and \lstinline{tgeometry}, this does not seem necessary yet, but if new geometry type are added and handled in the future, it might become more interesting.
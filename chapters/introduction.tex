%!TEX root = ../main.tex

\section{Context}

Temporal or spatio-temporal data exists in many types. It ranges from simple values evolving through time (ex. temperature) to positions or even regions moving through time. Most current relational databases are not well-suited to handle this temporal data, and thus heavy research has already been done to develop databases capable of storing and querying temporal data efficiently \cite{moving_obj_databases}. These databases are called \textit{Moving Object Databases} [MOD].

MODs have to handle a large variety of (spatio-)temporal object, also called \textit{moving objects}. Moving reals can be used to represent the temperature of a room; moving points store the position of cars, planes and more; moving regions are used to keep track of forest fires, glacier extents and more. Heavy research is being done on moving points, since this data is abundant and there is a need to analyze it efficiently. Due to the complexity and the limited use of moving regions, this area is less developed. Still, previous research analyzed the construction of a moving region from snapshots \cite{repr_from_obs,high_quality_interpol}, or developed novel data models for representing moving regions efficiently \cite{polyhedra,fmregion}.

Despite all this previous work, most research ideas still have to be implemented into a large open-source database to be available for everyone to use. \textit{MobilityDB} \cite{mobilitydb} aims to solve exactly that. It is an open-source extension on top of PostgreSQL \cite{postgresql} and PostGIS \cite{postgis} that adds spatio-temporal types for moving points as well as temporal types for moving booleans, integers, reals and strings. Implementation of moving region concepts into MobilityDB still has to be done, and this is the topic of this Master thesis.

\section{Structure and Objective of the Thesis}

The first part of this thesis describe the current implementation of moving objects into MobilityDB. This section details the structure of the system and the internal storage of the new types. The first goal of this thesis is to understand the system, since the next part will involve making additions and modifications.

In a second step, we will discuss the different type of moving regions, with examples of real-world use cases. Depending on the type of moving region that needs to be handled, the implementation can vary considerably. For this reason, this master thesis will focus mainly on one type, fixed-shape moving regions.

The next part concerns the implementation of fixed-shape moving regions into MobilityDB. The representation of moving regions will be discussed, as well as a set of operations that can be applied on this type. The main goal of the thesis is to have a working implementation of a specific type of moving region in MobilityDB, with a large set of functions implemented that can be used to process this data type. Implementation challenges, and their solutions, will be discussed for all implemented functions, and functions that remain unimplemented due to unsolved challenges will also be listed and described.

As a last step, this thesis will expand on the previously implemented types and functions, by providing ideas and algorithms to later be able to extend the system to handle a wider set of moving regions. The objective of this part is thus to already think about possible improvements to the systems, and define future work that could be done to be able to use the system with a wider set of moving regions, such as deforming or even three dimensional regions.

\section{Terms and Notations}

To make sure that there is no confusion or ambiguity, this section will define the terms and notations used in the rest of the thesis. Most of the terms will resemble the names already present in MobilityDB, since this is the primary system that will be worked on during this thesis.

\begin{itemize}
    \item This master thesis will mainly work on moving regions. A static region will be represented by the PostGIS Polygon type. For this reason, the terms \textit{region} and \textit{polygon} will be used interchangeably throughout this thesis.

    \item Moving regions can be categorized into \textit{deforming} regions and \textit{non-deforming} regions. The first paper \cite{fmregion} making this distinction uses the term \textit{fixed-shape} to denote these non-deforming regions. Based on this paper, other authors \cite{modeling_and_representing,polyhedra} also started to use this terminology, and we will thus do the same.

    \item Most research papers use the notation \textit{mbool}, \textit{mfloat}, ..., when talking about moving objects of respectively boolean or float base types. However, the notation in MobilityDB is different. There, moving objects are stored using \textit{temporal types}, and these types use a notation starting with a \textit{t}, followed by the base type. The two previous types thus become \textit{tbool} and \textit{tfloat}. This is the notation that will be used throughout this thesis.

    \item To denote moving points, again, most research papers use the notation \textit{mpoint}. In MobilityDB, the notation again uses a \textit{t} for temporal instead of an \textit{m} for moving. A distinction is also made between the base types \textit{geometry(Point)} and \textit{geography(Point)} from PostGIS, so the resulting notations are \textit{tgeompoint} and \textit{tgeogpoint}.

    \item Moving regions are usually written as \textit{mregion} in previous research, but the type is not present in MobilityDB yet, so the notation still has to be defined. In \cite{fmregion}, the distinction is made between regular moving regions \textit{mregion} and fixed-shape moving regions \textit{fmregion}. Since we will not discuss the implementation of regular moving regions, this distinction will not be made here. The distinction between the base types \textit{geometry(Polygon)} and \textit{geography(Polygon)}, however, will be made, since the operations are not always the same in both cases. For the rest of this master thesis, moving regions will be called \textit{tgeometry} and \textit{tgeography}, depending on their respective base type.

    \item Based on the three previous points, we will also use the terms \textit{moving} and \textit{temporal} interchangeably, to denote a value that changes though time.
\end{itemize}
Avec l'explosion de la quantité d'information disponible, les données spatiotemporelles deviennent une partie de plus en plus importante de notre société et ont de nombreuses applications pratiques. Cette utilisation accrue des données spatiotemporelles, également appelées objets en mouvement, alimente le besoin de bases de données spatiotemporelles accessibles à tous et capables de manipuler tout type d'objets en mouvement. Les points en mouvement étant l'exemple le plus utilisé, ils sont déjà traités dans les bases de données actuelles. Néanmoins, un autre domaine qui a déjà reçu une certaine attention dans la recherche, doit encore être traité en pratique pour pouvoir être utilisé dans le monde réel. Ce domaine concerne les régions en mouvement et fait l'objet de ce mémoire, qui décrit les différents types de régions en mouvement et la recherche actuelle dans ce domaine. L'objectif principal de ce mémoire est la mise en œuvre de régions mobiles de forme fixe dans MobilityDB, une base de données spatiotemporelle open source. À cette fin, un modèle pour représenter ce nouveau type de données est décrit et implémenté, ainsi que plusieurs fonctions et opérateurs pouvant être appliqués à ces régions en mouvement. Enfin, plusieurs extensions et possibles travaux futurs sont aussi présentés, et leur faisabilité et importance sont discutées. \\


\noindent\textbf{Mots clés:} données spatiotemporelles, base de données spatiotemporelle, MobilityDB, régions en mouvement, régions mobiles de forme fixe